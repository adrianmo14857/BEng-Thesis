\chapter{Wspęp}


Żaneta dzban 


Robotyka jest gałęzią technologii zajmującą się projektowaniem, budową, sterowaniem i użytkowaniem robotów. 

\section{Cel i założenia pracy}

Celem projektu dyplomowego było zaprojektowanie, zbudowanie oraz uruchomienie platformy dydaktycznej robota mobilnego, kołowego o napędzie elektrycznym, różnicowym. Napędy robota zostały zrealizowane za pomocą czterech silników DC, wraz z dedykowanym układem mocy Pololu VNH5019, posiadającym własny mikro kontroler oraz jednostkę inercyjną. Robot mobilny został wyposażony w sterownik główny typu: Intel Up-Board, wraz z dedykowanym systemem wizyjnym Intel RealSense R200. Sterownik robota został oprogramowany w jęzuku C/C++, był on odpowiedzialny za sterowanie napędami, czyli zadawanie oraz stabilizację prędkości, a także wyliczanie akualnej pozycj oraz parametrów ruchu na podstawie odczytow z enkoderów oraz jednostki inercyjnej. Informacje z platformy zostały udostępnione dla systemu nadrzędnego, pracującego pod kontrolą systmeu operacyjnego, wykrzystującego komunikację między wątkową ROS. Testy działania robota mobilnego zostały przeprowadzone z wykorzystaniem dedykowanych w robotyce mobilnej narzędzi programowych tj: ROS/RViz, Gazebo, MATLAB Robotics System Toolbox.

\section{Układ pracy}

Projekt dyplomowy składa się z sześciu rozdziałów. Rozdział pierwszy został poświęcony wprowadzeniu w tematykę robotyki oraz określeniu celów przed rozpoczęciem pracy nad projektem. W rozdziale drugim przybliżona została tematyka robotów mobilnych, komunikacja między wątkowa oraz platforma dydaktyczna Tutrlebot. W rozdziale trzecim został przedstawiona konstrukcja robota oraz charakterystyka podzespołów wykorzystanych do jego konstrukcji. Rozdział czwarty został poświęcony oprogramowaniu sterownika robota. W rozdziale piątym zostało opisane zastosowanie komunikacji między wątkowej. Rozdział szósty jest poświęcony testom działania robota mobilnego.
